
% Default to the notebook output style

    


% Inherit from the specified cell style.




    
\documentclass{article}

    
    
    \usepackage{graphicx} % Used to insert images
    \usepackage{adjustbox} % Used to constrain images to a maximum size 
    \usepackage{color} % Allow colors to be defined
    \usepackage{enumerate} % Needed for markdown enumerations to work
    \usepackage{geometry} % Used to adjust the document margins
    \usepackage{amsmath} % Equations
    \usepackage{amssymb} % Equations
    \usepackage{eurosym} % defines \euro
    \usepackage[mathletters]{ucs} % Extended unicode (utf-8) support
    \usepackage[utf8x]{inputenc} % Allow utf-8 characters in the tex document
    \usepackage{fancyvrb} % verbatim replacement that allows latex
    \usepackage{grffile} % extends the file name processing of package graphics 
                         % to support a larger range 
    % The hyperref package gives us a pdf with properly built
    % internal navigation ('pdf bookmarks' for the table of contents,
    % internal cross-reference links, web links for URLs, etc.)
    \usepackage{hyperref}
    \usepackage{longtable} % longtable support required by pandoc >1.10
    \usepackage{booktabs}  % table support for pandoc > 1.12.2
    

    % NG ADD ngarvey
    \pagenumbering{gobble} 
    \usepackage{parskip}
    
    \definecolor{orange}{cmyk}{0,0.4,0.8,0.2}
    \definecolor{darkorange}{rgb}{.71,0.21,0.01}
    \definecolor{darkgreen}{rgb}{.12,.54,.11}
    \definecolor{myteal}{rgb}{.26, .44, .56}
    \definecolor{gray}{gray}{0.45}
    \definecolor{lightgray}{gray}{.95}
    \definecolor{mediumgray}{gray}{.8}
    \definecolor{inputbackground}{rgb}{.95, .95, .85}
    \definecolor{outputbackground}{rgb}{.95, .95, .95}
    \definecolor{traceback}{rgb}{1, .95, .95}
    % ansi colors
    \definecolor{red}{rgb}{.6,0,0}
    \definecolor{green}{rgb}{0,.65,0}
    \definecolor{brown}{rgb}{0.6,0.6,0}
    \definecolor{blue}{rgb}{0,.145,.698}
    \definecolor{purple}{rgb}{.698,.145,.698}
    \definecolor{cyan}{rgb}{0,.698,.698}
    \definecolor{lightgray}{gray}{0.5}
    
    % bright ansi colors
    \definecolor{darkgray}{gray}{0.25}
    \definecolor{lightred}{rgb}{1.0,0.39,0.28}
    \definecolor{lightgreen}{rgb}{0.48,0.99,0.0}
    \definecolor{lightblue}{rgb}{0.53,0.81,0.92}
    \definecolor{lightpurple}{rgb}{0.87,0.63,0.87}
    \definecolor{lightcyan}{rgb}{0.5,1.0,0.83}
    
    % commands and environments needed by pandoc snippets
    % extracted from the output of `pandoc -s`
    \DefineVerbatimEnvironment{Highlighting}{Verbatim}{commandchars=\\\{\}}
    % Add ',fontsize=\small' for more characters per line
    \newenvironment{Shaded}{}{}
    \newcommand{\KeywordTok}[1]{\textcolor[rgb]{0.00,0.44,0.13}{\textbf{{#1}}}}
    \newcommand{\DataTypeTok}[1]{\textcolor[rgb]{0.56,0.13,0.00}{{#1}}}
    \newcommand{\DecValTok}[1]{\textcolor[rgb]{0.25,0.63,0.44}{{#1}}}
    \newcommand{\BaseNTok}[1]{\textcolor[rgb]{0.25,0.63,0.44}{{#1}}}
    \newcommand{\FloatTok}[1]{\textcolor[rgb]{0.25,0.63,0.44}{{#1}}}
    \newcommand{\CharTok}[1]{\textcolor[rgb]{0.25,0.44,0.63}{{#1}}}
    \newcommand{\StringTok}[1]{\textcolor[rgb]{0.25,0.44,0.63}{{#1}}}
    \newcommand{\CommentTok}[1]{\textcolor[rgb]{0.38,0.63,0.69}{\textit{{#1}}}}
    \newcommand{\OtherTok}[1]{\textcolor[rgb]{0.00,0.44,0.13}{{#1}}}
    \newcommand{\AlertTok}[1]{\textcolor[rgb]{1.00,0.00,0.00}{\textbf{{#1}}}}
    \newcommand{\FunctionTok}[1]{\textcolor[rgb]{0.02,0.16,0.49}{{#1}}}
    \newcommand{\RegionMarkerTok}[1]{{#1}}
    \newcommand{\ErrorTok}[1]{\textcolor[rgb]{1.00,0.00,0.00}{\textbf{{#1}}}}
    \newcommand{\NormalTok}[1]{{#1}}
    
    % Define a nice break command that doesn't care if a line doesn't already
    % exist.
    \def\br{\hspace*{\fill} \\* }
    % Math Jax compatability definitions
    \def\gt{>}
    \def\lt{<}
    % Document parameters
    \title{Project 1}
    
    
    

    % Pygments definitions
    
\makeatletter
\def\PY@reset{\let\PY@it=\relax \let\PY@bf=\relax%
    \let\PY@ul=\relax \let\PY@tc=\relax%
    \let\PY@bc=\relax \let\PY@ff=\relax}
\def\PY@tok#1{\csname PY@tok@#1\endcsname}
\def\PY@toks#1+{\ifx\relax#1\empty\else%
    \PY@tok{#1}\expandafter\PY@toks\fi}
\def\PY@do#1{\PY@bc{\PY@tc{\PY@ul{%
    \PY@it{\PY@bf{\PY@ff{#1}}}}}}}
\def\PY#1#2{\PY@reset\PY@toks#1+\relax+\PY@do{#2}}

\expandafter\def\csname PY@tok@gd\endcsname{\def\PY@tc##1{\textcolor[rgb]{0.63,0.00,0.00}{##1}}}
\expandafter\def\csname PY@tok@gu\endcsname{\let\PY@bf=\textbf\def\PY@tc##1{\textcolor[rgb]{0.50,0.00,0.50}{##1}}}
\expandafter\def\csname PY@tok@gt\endcsname{\def\PY@tc##1{\textcolor[rgb]{0.00,0.27,0.87}{##1}}}
\expandafter\def\csname PY@tok@gs\endcsname{\let\PY@bf=\textbf}
\expandafter\def\csname PY@tok@gr\endcsname{\def\PY@tc##1{\textcolor[rgb]{1.00,0.00,0.00}{##1}}}
\expandafter\def\csname PY@tok@cm\endcsname{\let\PY@it=\textit\def\PY@tc##1{\textcolor[rgb]{0.25,0.50,0.50}{##1}}}
\expandafter\def\csname PY@tok@vg\endcsname{\def\PY@tc##1{\textcolor[rgb]{0.10,0.09,0.49}{##1}}}
\expandafter\def\csname PY@tok@m\endcsname{\def\PY@tc##1{\textcolor[rgb]{0.40,0.40,0.40}{##1}}}
\expandafter\def\csname PY@tok@mh\endcsname{\def\PY@tc##1{\textcolor[rgb]{0.40,0.40,0.40}{##1}}}
\expandafter\def\csname PY@tok@go\endcsname{\def\PY@tc##1{\textcolor[rgb]{0.53,0.53,0.53}{##1}}}
\expandafter\def\csname PY@tok@ge\endcsname{\let\PY@it=\textit}
\expandafter\def\csname PY@tok@vc\endcsname{\def\PY@tc##1{\textcolor[rgb]{0.10,0.09,0.49}{##1}}}
\expandafter\def\csname PY@tok@il\endcsname{\def\PY@tc##1{\textcolor[rgb]{0.40,0.40,0.40}{##1}}}
\expandafter\def\csname PY@tok@cs\endcsname{\let\PY@it=\textit\def\PY@tc##1{\textcolor[rgb]{0.25,0.50,0.50}{##1}}}
\expandafter\def\csname PY@tok@cp\endcsname{\def\PY@tc##1{\textcolor[rgb]{0.74,0.48,0.00}{##1}}}
\expandafter\def\csname PY@tok@gi\endcsname{\def\PY@tc##1{\textcolor[rgb]{0.00,0.63,0.00}{##1}}}
\expandafter\def\csname PY@tok@gh\endcsname{\let\PY@bf=\textbf\def\PY@tc##1{\textcolor[rgb]{0.00,0.00,0.50}{##1}}}
\expandafter\def\csname PY@tok@ni\endcsname{\let\PY@bf=\textbf\def\PY@tc##1{\textcolor[rgb]{0.60,0.60,0.60}{##1}}}
\expandafter\def\csname PY@tok@nl\endcsname{\def\PY@tc##1{\textcolor[rgb]{0.63,0.63,0.00}{##1}}}
\expandafter\def\csname PY@tok@nn\endcsname{\let\PY@bf=\textbf\def\PY@tc##1{\textcolor[rgb]{0.00,0.00,1.00}{##1}}}
\expandafter\def\csname PY@tok@no\endcsname{\def\PY@tc##1{\textcolor[rgb]{0.53,0.00,0.00}{##1}}}
\expandafter\def\csname PY@tok@na\endcsname{\def\PY@tc##1{\textcolor[rgb]{0.49,0.56,0.16}{##1}}}
\expandafter\def\csname PY@tok@nb\endcsname{\def\PY@tc##1{\textcolor[rgb]{0.00,0.50,0.00}{##1}}}
\expandafter\def\csname PY@tok@nc\endcsname{\let\PY@bf=\textbf\def\PY@tc##1{\textcolor[rgb]{0.00,0.00,1.00}{##1}}}
\expandafter\def\csname PY@tok@nd\endcsname{\def\PY@tc##1{\textcolor[rgb]{0.67,0.13,1.00}{##1}}}
\expandafter\def\csname PY@tok@ne\endcsname{\let\PY@bf=\textbf\def\PY@tc##1{\textcolor[rgb]{0.82,0.25,0.23}{##1}}}
\expandafter\def\csname PY@tok@nf\endcsname{\def\PY@tc##1{\textcolor[rgb]{0.00,0.00,1.00}{##1}}}
\expandafter\def\csname PY@tok@si\endcsname{\let\PY@bf=\textbf\def\PY@tc##1{\textcolor[rgb]{0.73,0.40,0.53}{##1}}}
\expandafter\def\csname PY@tok@s2\endcsname{\def\PY@tc##1{\textcolor[rgb]{0.73,0.13,0.13}{##1}}}
\expandafter\def\csname PY@tok@vi\endcsname{\def\PY@tc##1{\textcolor[rgb]{0.10,0.09,0.49}{##1}}}
\expandafter\def\csname PY@tok@nt\endcsname{\let\PY@bf=\textbf\def\PY@tc##1{\textcolor[rgb]{0.00,0.50,0.00}{##1}}}
\expandafter\def\csname PY@tok@nv\endcsname{\def\PY@tc##1{\textcolor[rgb]{0.10,0.09,0.49}{##1}}}
\expandafter\def\csname PY@tok@s1\endcsname{\def\PY@tc##1{\textcolor[rgb]{0.73,0.13,0.13}{##1}}}
\expandafter\def\csname PY@tok@kd\endcsname{\let\PY@bf=\textbf\def\PY@tc##1{\textcolor[rgb]{0.00,0.50,0.00}{##1}}}
\expandafter\def\csname PY@tok@sh\endcsname{\def\PY@tc##1{\textcolor[rgb]{0.73,0.13,0.13}{##1}}}
\expandafter\def\csname PY@tok@sc\endcsname{\def\PY@tc##1{\textcolor[rgb]{0.73,0.13,0.13}{##1}}}
\expandafter\def\csname PY@tok@sx\endcsname{\def\PY@tc##1{\textcolor[rgb]{0.00,0.50,0.00}{##1}}}
\expandafter\def\csname PY@tok@bp\endcsname{\def\PY@tc##1{\textcolor[rgb]{0.00,0.50,0.00}{##1}}}
\expandafter\def\csname PY@tok@c1\endcsname{\let\PY@it=\textit\def\PY@tc##1{\textcolor[rgb]{0.25,0.50,0.50}{##1}}}
\expandafter\def\csname PY@tok@kc\endcsname{\let\PY@bf=\textbf\def\PY@tc##1{\textcolor[rgb]{0.00,0.50,0.00}{##1}}}
\expandafter\def\csname PY@tok@c\endcsname{\let\PY@it=\textit\def\PY@tc##1{\textcolor[rgb]{0.25,0.50,0.50}{##1}}}
\expandafter\def\csname PY@tok@mf\endcsname{\def\PY@tc##1{\textcolor[rgb]{0.40,0.40,0.40}{##1}}}
\expandafter\def\csname PY@tok@err\endcsname{\def\PY@bc##1{\setlength{\fboxsep}{0pt}\fcolorbox[rgb]{1.00,0.00,0.00}{1,1,1}{\strut ##1}}}
\expandafter\def\csname PY@tok@mb\endcsname{\def\PY@tc##1{\textcolor[rgb]{0.40,0.40,0.40}{##1}}}
\expandafter\def\csname PY@tok@ss\endcsname{\def\PY@tc##1{\textcolor[rgb]{0.10,0.09,0.49}{##1}}}
\expandafter\def\csname PY@tok@sr\endcsname{\def\PY@tc##1{\textcolor[rgb]{0.73,0.40,0.53}{##1}}}
\expandafter\def\csname PY@tok@mo\endcsname{\def\PY@tc##1{\textcolor[rgb]{0.40,0.40,0.40}{##1}}}
\expandafter\def\csname PY@tok@kn\endcsname{\let\PY@bf=\textbf\def\PY@tc##1{\textcolor[rgb]{0.00,0.50,0.00}{##1}}}
\expandafter\def\csname PY@tok@mi\endcsname{\def\PY@tc##1{\textcolor[rgb]{0.40,0.40,0.40}{##1}}}
\expandafter\def\csname PY@tok@gp\endcsname{\let\PY@bf=\textbf\def\PY@tc##1{\textcolor[rgb]{0.00,0.00,0.50}{##1}}}
\expandafter\def\csname PY@tok@o\endcsname{\def\PY@tc##1{\textcolor[rgb]{0.40,0.40,0.40}{##1}}}
\expandafter\def\csname PY@tok@kr\endcsname{\let\PY@bf=\textbf\def\PY@tc##1{\textcolor[rgb]{0.00,0.50,0.00}{##1}}}
\expandafter\def\csname PY@tok@s\endcsname{\def\PY@tc##1{\textcolor[rgb]{0.73,0.13,0.13}{##1}}}
\expandafter\def\csname PY@tok@kp\endcsname{\def\PY@tc##1{\textcolor[rgb]{0.00,0.50,0.00}{##1}}}
\expandafter\def\csname PY@tok@w\endcsname{\def\PY@tc##1{\textcolor[rgb]{0.73,0.73,0.73}{##1}}}
\expandafter\def\csname PY@tok@kt\endcsname{\def\PY@tc##1{\textcolor[rgb]{0.69,0.00,0.25}{##1}}}
\expandafter\def\csname PY@tok@ow\endcsname{\let\PY@bf=\textbf\def\PY@tc##1{\textcolor[rgb]{0.67,0.13,1.00}{##1}}}
\expandafter\def\csname PY@tok@sb\endcsname{\def\PY@tc##1{\textcolor[rgb]{0.73,0.13,0.13}{##1}}}
\expandafter\def\csname PY@tok@k\endcsname{\let\PY@bf=\textbf\def\PY@tc##1{\textcolor[rgb]{0.00,0.50,0.00}{##1}}}
\expandafter\def\csname PY@tok@se\endcsname{\let\PY@bf=\textbf\def\PY@tc##1{\textcolor[rgb]{0.73,0.40,0.13}{##1}}}
\expandafter\def\csname PY@tok@sd\endcsname{\let\PY@it=\textit\def\PY@tc##1{\textcolor[rgb]{0.73,0.13,0.13}{##1}}}

\def\PYZbs{\char`\\}
\def\PYZus{\char`\_}
\def\PYZob{\char`\{}
\def\PYZcb{\char`\}}
\def\PYZca{\char`\^}
\def\PYZam{\char`\&}
\def\PYZlt{\char`\<}
\def\PYZgt{\char`\>}
\def\PYZsh{\char`\#}
\def\PYZpc{\char`\%}
\def\PYZdl{\char`\$}
\def\PYZhy{\char`\-}
\def\PYZsq{\char`\'}
\def\PYZdq{\char`\"}
\def\PYZti{\char`\~}
% for compatibility with earlier versions
\def\PYZat{@}
\def\PYZlb{[}
\def\PYZrb{]}
\makeatother


    % Exact colors from NB
    \definecolor{incolor}{rgb}{0.0, 0.0, 0.5}
    \definecolor{outcolor}{rgb}{0.545, 0.0, 0.0}



    
    % Prevent overflowing lines due to hard-to-break entities
    \sloppy 
    % Setup hyperref package
    \hypersetup{
      breaklinks=true,  % so long urls are correctly broken across lines
      colorlinks=true,
      urlcolor=blue,
      linkcolor=darkorange,
      citecolor=darkgreen,
      }
    % Slightly bigger margins than the latex defaults
    
    \geometry{verbose,tmargin=1in,bmargin=1in,lmargin=1in,rmargin=1in}
    
    

    \begin{document}
    
    
    \maketitle
    
    

    
    \section*{Effect of Rain on Subway Ridership on
Weekdays}\label{effect-of-rain-on-subway-ridership-on-weekdays}

\section*{Nick Garvey}\label{nick-garvey}

    \begin{Verbatim}[commandchars=\\\{\}]
{\color{incolor}In [{\color{incolor}3}]:} \PY{k+kn}{from} \PY{n+nn}{\PYZus{}\PYZus{}future\PYZus{}\PYZus{}} \PY{k+kn}{import} \PY{n}{absolute\PYZus{}import}
        \PY{k+kn}{from} \PY{n+nn}{\PYZus{}\PYZus{}future\PYZus{}\PYZus{}} \PY{k+kn}{import} \PY{n}{division}
        \PY{k+kn}{from} \PY{n+nn}{\PYZus{}\PYZus{}future\PYZus{}\PYZus{}} \PY{k+kn}{import} \PY{n}{print\PYZus{}function}
        \PY{k+kn}{from} \PY{n+nn}{\PYZus{}\PYZus{}future\PYZus{}\PYZus{}} \PY{k+kn}{import} \PY{n}{unicode\PYZus{}literals}
        
        \PY{k+kn}{import} \PY{n+nn}{pandas} \PY{k+kn}{as} \PY{n+nn}{pd}
        \PY{k+kn}{import} \PY{n+nn}{numpy} \PY{k+kn}{as} \PY{n+nn}{np}
        \PY{k+kn}{import} \PY{n+nn}{scipy} \PY{k+kn}{as} \PY{n+nn}{sp}
        \PY{k+kn}{import} \PY{n+nn}{ggplot} \PY{k+kn}{as} \PY{n+nn}{gp}
        \PY{k+kn}{import} \PY{n+nn}{matplotlib.pyplot} \PY{k+kn}{as} \PY{n+nn}{plt}
        \PY{k+kn}{import} \PY{n+nn}{matplotlib.mlab} \PY{k+kn}{as} \PY{n+nn}{mlab}
        \PY{k+kn}{import} \PY{n+nn}{statsmodels.formula.api} \PY{k+kn}{as} \PY{n+nn}{sm}
        
        \PY{o}{\PYZpc{}}\PY{k}{matplotlib} inline
\end{Verbatim}

    \begin{Verbatim}[commandchars=\\\{\}]
{\color{incolor}In [{\color{incolor}4}]:} \PY{c}{\PYZsh{} Data Source: https://www.dropbox.com/s/}
        \PY{c}{\PYZsh{} meyki2wl9xfa7yk/turnstile\PYZus{}data\PYZus{}master\PYZus{}with\PYZus{}weather.csv}
        
        \PY{n}{df} \PY{o}{=} \PY{n}{pd}\PY{o}{.}\PY{n}{read\PYZus{}csv}\PY{p}{(}\PY{l+s}{\PYZsq{}}\PY{l+s}{turnstile\PYZus{}weather\PYZus{}v2.csv}\PY{l+s}{\PYZsq{}}\PY{p}{)}
\end{Verbatim}

    \subsection*{0. References}\label{references}

    \begin{itemize}
\itemsep1pt\parskip0pt\parsep0pt
\item
  Python for Data Analysis: Data Wrangling with Pandas, NumPy, and
  IPython by Wes McKinney
\item
  Share label between subplots
  http://stackoverflow.com/a/26892326/965648
\item
  Haversine formula details
  http://en.wikipedia.org/wiki/Haversine\_formula
\end{itemize}

    \subsection*{1. Statistical Test}\label{statistical-test}

    First, check if the data is normal by graphing it.

    \begin{Verbatim}[commandchars=\\\{\}]
{\color{incolor}In [{\color{incolor}5}]:} \PY{c}{\PYZsh{} Monday is 0, ... Saturday is 5, Sunday is 6}
        \PY{n}{week\PYZus{}days} \PY{o}{=} \PY{n}{df}\PY{p}{[}\PY{n}{pd}\PY{o}{.}\PY{n}{to\PYZus{}datetime}\PY{p}{(}\PY{n}{df}\PY{p}{[}\PY{l+s}{\PYZsq{}}\PY{l+s}{DATEn}\PY{l+s}{\PYZsq{}}\PY{p}{]}\PY{p}{)}\PY{o}{.}\PY{n}{dt}\PY{o}{.}\PY{n}{dayofweek} \PY{o}{\PYZlt{}} \PY{l+m+mi}{5}\PY{p}{]}
        \PY{n}{week\PYZus{}days} \PY{o}{=} \PY{n}{week\PYZus{}days}\PY{p}{[}\PY{p}{[}\PY{l+s}{\PYZsq{}}\PY{l+s}{station}\PY{l+s}{\PYZsq{}}\PY{p}{,} \PY{l+s}{\PYZsq{}}\PY{l+s}{DATEn}\PY{l+s}{\PYZsq{}}\PY{p}{,} \PY{l+s}{\PYZsq{}}\PY{l+s}{rain}\PY{l+s}{\PYZsq{}}\PY{p}{,} \PY{l+s}{\PYZsq{}}\PY{l+s}{ENTRIESn\PYZus{}hourly}\PY{l+s}{\PYZsq{}}\PY{p}{]}\PY{p}{]}
        \PY{n}{grouped\PYZus{}days} \PY{o}{=} \PY{n}{week\PYZus{}days}\PY{o}{.}\PY{n}{groupby}\PY{p}{(}\PY{p}{[}\PY{l+s}{\PYZsq{}}\PY{l+s}{station}\PY{l+s}{\PYZsq{}}\PY{p}{,} \PY{l+s}{\PYZsq{}}\PY{l+s}{DATEn}\PY{l+s}{\PYZsq{}}\PY{p}{]}\PY{p}{)}
        
        \PY{c}{\PYZsh{} Add a column, rainy, that annotates if it was rainy at that station at all during that day}
        \PY{k}{def} \PY{n+nf}{add\PYZus{}rainy\PYZus{}day}\PY{p}{(}\PY{n}{group}\PY{p}{)}\PY{p}{:}
            \PY{n}{group}\PY{p}{[}\PY{l+s}{\PYZsq{}}\PY{l+s}{rainy\PYZus{}day}\PY{l+s}{\PYZsq{}}\PY{p}{]} \PY{o}{=} \PY{l+m+mi}{1} \PY{o+ow}{in} \PY{n}{group}\PY{p}{[}\PY{l+s}{\PYZsq{}}\PY{l+s}{rain}\PY{l+s}{\PYZsq{}}\PY{p}{]}\PY{o}{.}\PY{n}{values}
            \PY{k}{return} \PY{n}{group}
        
        \PY{n}{df\PYZus{}with\PYZus{}rainy} \PY{o}{=} \PY{n}{grouped\PYZus{}days}\PY{o}{.}\PY{n}{apply}\PY{p}{(}\PY{n}{add\PYZus{}rainy\PYZus{}day}\PY{p}{)}
        \PY{n}{by\PYZus{}day} \PY{o}{=} \PY{n}{df\PYZus{}with\PYZus{}rainy}\PY{o}{.}\PY{n}{groupby}\PY{p}{(}\PY{p}{[}\PY{l+s}{\PYZsq{}}\PY{l+s}{DATEn}\PY{l+s}{\PYZsq{}}\PY{p}{,} \PY{l+s}{\PYZsq{}}\PY{l+s}{station}\PY{l+s}{\PYZsq{}}\PY{p}{,} \PY{l+s}{\PYZsq{}}\PY{l+s}{rainy\PYZus{}day}\PY{l+s}{\PYZsq{}}\PY{p}{]}\PY{p}{,} \PY{n}{as\PYZus{}index}\PY{o}{=}\PY{n+nb+bp}{False}\PY{p}{)}
        \PY{n}{by\PYZus{}day} \PY{o}{=} \PY{n}{by\PYZus{}day}\PY{p}{[}\PY{l+s}{\PYZsq{}}\PY{l+s}{ENTRIESn\PYZus{}hourly}\PY{l+s}{\PYZsq{}}\PY{p}{]}\PY{o}{.}\PY{n}{sum}\PY{p}{(}\PY{p}{)}
        
        \PY{n}{rain} \PY{o}{=} \PY{n}{by\PYZus{}day}\PY{p}{[}\PY{n}{by\PYZus{}day}\PY{p}{[}\PY{l+s}{\PYZsq{}}\PY{l+s}{rainy\PYZus{}day}\PY{l+s}{\PYZsq{}}\PY{p}{]} \PY{o}{==} \PY{n+nb+bp}{True}\PY{p}{]}\PY{p}{[}\PY{l+s}{\PYZsq{}}\PY{l+s}{ENTRIESn\PYZus{}hourly}\PY{l+s}{\PYZsq{}}\PY{p}{]}
        \PY{n}{no\PYZus{}rain} \PY{o}{=} \PY{n}{by\PYZus{}day}\PY{p}{[}\PY{n}{by\PYZus{}day}\PY{p}{[}\PY{l+s}{\PYZsq{}}\PY{l+s}{rainy\PYZus{}day}\PY{l+s}{\PYZsq{}}\PY{p}{]} \PY{o}{==} \PY{n+nb+bp}{False}\PY{p}{]}\PY{p}{[}\PY{l+s}{\PYZsq{}}\PY{l+s}{ENTRIESn\PYZus{}hourly}\PY{l+s}{\PYZsq{}}\PY{p}{]}
\end{Verbatim}

    \begin{Verbatim}[commandchars=\\\{\}]
{\color{incolor}In [{\color{incolor}6}]:} \PY{n}{fig}\PY{p}{,} \PY{n}{axes} \PY{o}{=} \PY{n}{plt}\PY{o}{.}\PY{n}{subplots}\PY{p}{(}\PY{l+m+mi}{2}\PY{p}{,} \PY{n}{sharex}\PY{o}{=}\PY{n+nb+bp}{True}\PY{p}{)}
        
        \PY{n}{no\PYZus{}rain}\PY{o}{.}\PY{n}{hist}\PY{p}{(}\PY{n}{ax}\PY{o}{=}\PY{n}{axes}\PY{p}{[}\PY{l+m+mi}{0}\PY{p}{]}\PY{p}{,} \PY{n}{bins}\PY{o}{=}\PY{l+m+mi}{20}\PY{p}{)}\PY{o}{.}\PY{n}{set\PYZus{}title}\PY{p}{(}\PY{l+s}{\PYZsq{}}\PY{l+s}{Weekday Rainless Subway Station Attendance}\PY{l+s}{\PYZsq{}}\PY{p}{)}
        \PY{n}{rain}\PY{o}{.}\PY{n}{hist}\PY{p}{(}\PY{n}{ax}\PY{o}{=}\PY{n}{axes}\PY{p}{[}\PY{l+m+mi}{1}\PY{p}{]}\PY{p}{,} \PY{n}{bins}\PY{o}{=}\PY{l+m+mi}{20}\PY{p}{)}\PY{o}{.}\PY{n}{set\PYZus{}title}\PY{p}{(}\PY{l+s}{\PYZsq{}}\PY{l+s}{Weekday Rainy Subway Station Attendance}\PY{l+s}{\PYZsq{}}\PY{p}{)}
        \PY{n}{fig}\PY{o}{.}\PY{n}{subplots\PYZus{}adjust}\PY{p}{(}\PY{n}{hspace}\PY{o}{=}\PY{o}{.}\PY{l+m+mi}{45}\PY{p}{)}
        
        \PY{c}{\PYZsh{} Apply the labels just as text as it isn\PYZsq{}t obvious how to share a label among two subplots}
        \PY{c}{\PYZsh{} Code from http://stackoverflow.com/a/26892326/965648}
        \PY{n}{fig}\PY{o}{.}\PY{n}{text}\PY{p}{(}\PY{l+m+mf}{0.5}\PY{p}{,} \PY{l+m+mf}{0.02}\PY{p}{,} \PY{l+s}{\PYZsq{}}\PY{l+s}{Daily Subway Entries Per Station}\PY{l+s}{\PYZsq{}}\PY{p}{,} \PY{n}{ha}\PY{o}{=}\PY{l+s}{\PYZsq{}}\PY{l+s}{center}\PY{l+s}{\PYZsq{}}\PY{p}{)}
        \PY{n}{fig}\PY{o}{.}\PY{n}{text}\PY{p}{(}\PY{l+m+mf}{0.02}\PY{p}{,} \PY{l+m+mf}{0.5}\PY{p}{,} \PY{l+s}{\PYZsq{}}\PY{l+s}{Number of Days Per Station}\PY{l+s}{\PYZsq{}}\PY{p}{,} \PY{n}{va}\PY{o}{=}\PY{l+s}{\PYZsq{}}\PY{l+s}{center}\PY{l+s}{\PYZsq{}}\PY{p}{,} \PY{n}{rotation}\PY{o}{=}\PY{l+s}{\PYZsq{}}\PY{l+s}{vertical}\PY{l+s}{\PYZsq{}}\PY{p}{)}
        
        \PY{n}{plt}\PY{o}{.}\PY{n}{show}\PY{p}{(}\PY{p}{)}
\end{Verbatim}

    \begin{center}
    \adjustimage{max size={0.9\linewidth}{0.9\paperheight}}{Project 1_files/Project 1_8_0.png}
    \end{center}
    { \hspace*{\fill} \\}
    
    \emph{1.1 Which statistical test did you use to analyze the NYC subway
data? Did you use a one-tail or a two-tail P value? What is the null
hypothesis? What is your p-critical value?}

The two-tailed Mann-Whitney U Test is used to compare the two samples.
We will use a p-critical value of .05.

Null hypothesis: The hourly entry rate is the same on rainy and
non-rainy days. Alternative hypothesis: The hourly entry rate is
different on rainy and non-rainy days.

\emph{1.2 Why is this statistical test applicable to the dataset? In
particular, consider the assumptions that the test is making about the
distribution of ridership in the two samples.}

Data is clearly not normal, so the Mann-Whitney U Test was chosen
instead of a t-test.

\emph{1.3 What results did you get from this statistical test? These
should include the following numerical values: p-values, as well as the
means for each of the two samples under test.}

See results below.

    \begin{Verbatim}[commandchars=\\\{\}]
{\color{incolor}In [{\color{incolor}11}]:} \PY{n}{U}\PY{p}{,} \PY{n}{p} \PY{o}{=} \PY{n}{sp}\PY{o}{.}\PY{n}{stats}\PY{o}{.}\PY{n}{mannwhitneyu}\PY{p}{(}\PY{n}{rain}\PY{p}{,} \PY{n}{no\PYZus{}rain}\PY{p}{)}
         
         \PY{k}{print}\PY{p}{(}\PY{l+s}{\PYZsq{}}\PY{l+s}{Mean hourly entries on a rainy day:}\PY{l+s}{\PYZsq{}}\PY{p}{,} \PY{n}{rain}\PY{o}{.}\PY{n}{mean}\PY{p}{(}\PY{p}{)}\PY{p}{)}
         \PY{k}{print}\PY{p}{(}\PY{l+s}{\PYZsq{}}\PY{l+s}{Mean hourly entries on a non\PYZhy{}rainy day:}\PY{l+s}{\PYZsq{}}\PY{p}{,} \PY{n}{no\PYZus{}rain}\PY{o}{.}\PY{n}{mean}\PY{p}{(}\PY{p}{)}\PY{p}{)}
         \PY{k}{print}\PY{p}{(}\PY{l+s}{\PYZsq{}}\PY{l+s}{U statistic:}\PY{l+s}{\PYZsq{}}\PY{p}{,} \PY{n}{U}\PY{p}{)}
         \PY{k}{print}\PY{p}{(}\PY{l+s}{\PYZsq{}}\PY{l+s}{p\PYZhy{}value:}\PY{l+s}{\PYZsq{}}\PY{p}{,} \PY{n}{p}\PY{p}{)}
\end{Verbatim}

    \begin{Verbatim}[commandchars=\\\{\}]
Mean hourly entries on a rainy day: 15091.9598291
Mean hourly entries on a non-rainy day: 14280.8696556
U statistic: 1914829.0
p-value: 0.0754622901355
    \end{Verbatim}

    \emph{1.4 What is the significance and interpretation of these results?}

Due to the P value of 7.5\%, we fail to reject the null.

    \subsection*{2. Linear Regression}\label{linear-regression}

    \emph{2.1 What approach did you use to compute the coefficients theta
and produce prediction for ENTRIESn\_hourly in your regression model}

Statsmodel OLS through the Pandas interface.

    \begin{Verbatim}[commandchars=\\\{\}]
{\color{incolor}In [{\color{incolor}12}]:} \PY{c}{\PYZsh{} Produce a dataset that groups all of the entries at a particular time together}
         \PY{c}{\PYZsh{} This way, the station/unit isn\PYZsq{}t considered}
         
         \PY{n}{total\PYZus{}hourly\PYZus{}entries} \PY{o}{=} \PYZbs{}
             \PY{n}{df}\PY{o}{.}\PY{n}{groupby}\PY{p}{(}\PY{p}{[}\PY{l+s}{\PYZsq{}}\PY{l+s}{weekday}\PY{l+s}{\PYZsq{}}\PY{p}{,} \PY{l+s}{\PYZsq{}}\PY{l+s}{hour}\PY{l+s}{\PYZsq{}}\PY{p}{,} \PY{l+s}{\PYZsq{}}\PY{l+s}{rain}\PY{l+s}{\PYZsq{}}\PY{p}{]}\PY{p}{,} \PY{n}{as\PYZus{}index}\PY{o}{=}\PY{n+nb+bp}{False}\PY{p}{)}\PY{p}{[}\PY{p}{[}\PY{l+s}{\PYZsq{}}\PY{l+s}{ENTRIESn\PYZus{}hourly}\PY{l+s}{\PYZsq{}}\PY{p}{]}\PY{p}{]}\PY{o}{.}\PY{n}{mean}\PY{p}{(}\PY{p}{)}
         
         \PY{n}{y} \PY{o}{=} \PY{n}{total\PYZus{}hourly\PYZus{}entries}\PY{p}{[}\PY{l+s}{\PYZsq{}}\PY{l+s}{ENTRIESn\PYZus{}hourly}\PY{l+s}{\PYZsq{}}\PY{p}{]}
         \PY{n}{x} \PY{o}{=} \PY{n}{total\PYZus{}hourly\PYZus{}entries}\PY{p}{[}\PY{p}{[}\PY{l+s}{\PYZsq{}}\PY{l+s}{hour}\PY{l+s}{\PYZsq{}}\PY{p}{,} \PY{l+s}{\PYZsq{}}\PY{l+s}{weekday}\PY{l+s}{\PYZsq{}}\PY{p}{,} \PY{l+s}{\PYZsq{}}\PY{l+s}{rain}\PY{l+s}{\PYZsq{}}\PY{p}{]}\PY{p}{]}
         \PY{c}{\PYZsh{} This is just statsmodel OLS, but we use the pandas interface to it}
         \PY{k}{print}\PY{p}{(}\PY{n}{pd}\PY{o}{.}\PY{n}{stats}\PY{o}{.}\PY{n}{ols}\PY{o}{.}\PY{n}{OLS}\PY{p}{(}\PY{n}{y}\PY{p}{,} \PY{n}{x}\PY{p}{)}\PY{p}{)}
\end{Verbatim}

    \begin{Verbatim}[commandchars=\\\{\}]
-------------------------Summary of Regression Analysis-------------------------

Formula: Y \textasciitilde{} <hour> + <weekday> + <rain> + <intercept>

Number of Observations:         24
Number of Degrees of Freedom:   4

R-squared:         0.5588
Adj R-squared:     0.4926

Rmse:            844.0648

F-stat (3, 20):     8.4422, p-value:     0.0008

Degrees of Freedom: model 3, resid 20

-----------------------Summary of Estimated Coefficients------------------------
      Variable       Coef    Std Err     t-stat    p-value    CI 2.5\%   CI 97.5\%
--------------------------------------------------------------------------------
          hour   101.2211    25.2213       4.01     0.0007    51.7874   150.6547
       weekday  1046.1914   344.5880       3.04     0.0065   370.7989  1721.5838
          rain   -15.4076   344.5880      -0.04     0.9648  -690.8001   659.9848
     intercept   108.0618   390.7261       0.28     0.7850  -657.7613   873.8848
---------------------------------End of Summary---------------------------------
    \end{Verbatim}

    \emph{2.2 What features (input variables) did you use in your model? Did
you use any dummy variables as part of your features?}

Input variables were weekday, hour, and rain. Rain was a dummy variable.

\emph{2.3 Why did you select these features in your model? We are
looking for specific reasons that lead you to believe that the selected
features will contribute to the predictive power of your model.}

The regression was done using hour, weekday, and rain as independent
variables. Hour and weekday were chosen as habit and routine are the
most likely significant driver of subway usage. Rain was included as it
was relevant to the assignment.

\emph{2.4 What are the coefficients (or weights) of the non-dummy
features in your linear regression model?}

The hour coefficient was 101.2211 and the weekday coefficient was
1046.1914

\emph{2.5 What is your model's R2 (coefficients of determination)
value?}

The R2 value is 0.5588.

\emph{2.6 What does this R2 value mean for the goodness of fit for your
regression model? Do you think this linear model to predict ridership is
appropriate for this dataset, given this R2 value?}

The above ordinary least squares analysis shows that 55.2\% of the
variance in the global hourly subway entries is explained by the hour
and if it is a weekday. This suggests a good relationship between if it
is weekday, the time of day, and the number of people that enter the
subway in New York City at that time.

    \subsection*{3. Visualization}\label{visualization}

    \emph{3.1 One visualization should contain two histograms: one of
ENTRIESn\_hourly for rainy days and one of ENTRIESn\_hourly for
non-rainy days.}

This was included above. It was used to justify the Mann-Whitney U Test,
so it was more appropriate in that section.

    \emph{3.2 One visualization can be more freeform. You should feel free
to implement something that we discussed in class (e.g., scatter plots,
line plots) or attempt to implement something more advanced if you'd
like}

    \begin{Verbatim}[commandchars=\\\{\}]
{\color{incolor}In [{\color{incolor}9}]:} \PY{c}{\PYZsh{} Our goal is to show how close the weather stations are to the subway stations that they provide}
        \PY{c}{\PYZsh{} weather data for. To do this, we use the latitude and longitude of each to determine where they are}
        \PY{c}{\PYZsh{} on the Earth. From these points, we use the Haversine formula to calculate the distance. While this}
        \PY{c}{\PYZsh{} formula assumes the Earth is a sphere, it should be more than good enough for our purposes}
        
        \PY{c}{\PYZsh{} Formula from http://en.wikipedia.org/wiki/Haversine\PYZus{}formula}
        \PY{c}{\PYZsh{} Implementation my own}
        \PY{k}{def} \PY{n+nf}{haversine}\PY{p}{(}\PY{n}{radius}\PY{p}{,} \PY{n}{lat1}\PY{p}{,} \PY{n}{lat2}\PY{p}{,} \PY{n}{lon1}\PY{p}{,} \PY{n}{lon2}\PY{p}{)}\PY{p}{:}
            \PY{l+s+sd}{\PYZdq{}\PYZdq{}\PYZdq{}Haversine formula. All angle variables are in radians\PYZdq{}\PYZdq{}\PYZdq{}}
            \PY{n}{inner} \PY{o}{=} \PY{n}{np}\PY{o}{.}\PY{n}{sqrt}\PY{p}{(}\PY{n}{np}\PY{o}{.}\PY{n}{sin}\PY{p}{(}\PY{p}{(}\PY{n}{lat2} \PY{o}{\PYZhy{}} \PY{n}{lat1}\PY{p}{)}\PY{o}{/}\PY{l+m+mi}{2}\PY{p}{)}\PY{o}{*}\PY{o}{*}\PY{l+m+mi}{2}
                            \PY{o}{+} \PY{p}{(}\PY{n}{np}\PY{o}{.}\PY{n}{cos}\PY{p}{(}\PY{n}{lat1}\PY{p}{)} \PY{o}{*} \PY{n}{np}\PY{o}{.}\PY{n}{cos}\PY{p}{(}\PY{n}{lat2}\PY{p}{)} \PY{o}{*} \PY{n}{np}\PY{o}{.}\PY{n}{sin}\PY{p}{(}\PY{p}{(}\PY{n}{lon2} \PY{o}{\PYZhy{}} \PY{n}{lon1}\PY{p}{)}\PY{o}{/}\PY{l+m+mi}{2}\PY{p}{)}\PY{o}{*}\PY{o}{*}\PY{l+m+mi}{2}\PY{p}{)}\PY{p}{)}
            \PY{k}{return} \PY{l+m+mi}{2} \PY{o}{*} \PY{n}{radius} \PY{o}{*} \PY{n}{np}\PY{o}{.}\PY{n}{arcsin}\PY{p}{(}\PY{n}{inner}\PY{p}{)}
        
        \PY{c}{\PYZsh{} This function was tested by plugging some values into this and}
        \PY{c}{\PYZsh{} comparing with Google Maps walking distance}
        \PY{k}{def} \PY{n+nf}{earth\PYZus{}distance}\PY{p}{(}\PY{n}{row}\PY{p}{)}\PY{p}{:}
            \PY{l+s+sd}{\PYZdq{}\PYZdq{}\PYZdq{}Variables are GPS coordinates\PYZdq{}\PYZdq{}\PYZdq{}}
            \PY{c}{\PYZsh{} Average Earth radius}
            \PY{n}{EARTH\PYZus{}RADIUS\PYZus{}MILES} \PY{o}{=} \PY{l+m+mi}{3959}
            \PY{k}{return} \PY{n}{haversine}\PY{p}{(}\PY{n}{EARTH\PYZus{}RADIUS\PYZus{}MILES}\PY{p}{,} \PY{o}{*}\PY{n}{np}\PY{o}{.}\PY{n}{deg2rad}\PY{p}{(}\PY{n}{row}\PY{p}{)}\PY{p}{)}
        
        \PY{n}{coords} \PY{o}{=} \PY{n}{df}\PY{o}{.}\PY{n}{groupby}\PY{p}{(}\PY{l+s}{\PYZsq{}}\PY{l+s}{station}\PY{l+s}{\PYZsq{}}\PY{p}{)}\PY{p}{[}\PY{p}{[}\PY{l+s}{\PYZsq{}}\PY{l+s}{latitude}\PY{l+s}{\PYZsq{}}\PY{p}{,} \PY{l+s}{\PYZsq{}}\PY{l+s}{weather\PYZus{}lat}\PY{l+s}{\PYZsq{}}\PY{p}{,} \PY{l+s}{\PYZsq{}}\PY{l+s}{longitude}\PY{l+s}{\PYZsq{}}\PY{p}{,} \PY{l+s}{\PYZsq{}}\PY{l+s}{weather\PYZus{}lon}\PY{l+s}{\PYZsq{}}\PY{p}{]}\PY{p}{]}
        \PY{n}{per\PYZus{}station\PYZus{}coords} \PY{o}{=} \PY{n}{coords}\PY{o}{.}\PY{n}{last}\PY{p}{(}\PY{p}{)}
        \PY{n}{per\PYZus{}station\PYZus{}coords}\PY{p}{[}\PY{l+s}{\PYZsq{}}\PY{l+s}{dist}\PY{l+s}{\PYZsq{}}\PY{p}{]} \PY{o}{=} \PYZbs{}
            \PY{n}{np}\PY{o}{.}\PY{n}{apply\PYZus{}along\PYZus{}axis}\PY{p}{(}\PY{n}{earth\PYZus{}distance}\PY{p}{,} \PY{l+m+mi}{1}\PY{p}{,} \PY{n}{per\PYZus{}station\PYZus{}coords}\PY{o}{.}\PY{n}{values}\PY{p}{)}
        
        \PY{n}{axis} \PY{o}{=} \PY{n}{per\PYZus{}station\PYZus{}coords}\PY{p}{[}\PY{l+s}{\PYZsq{}}\PY{l+s}{dist}\PY{l+s}{\PYZsq{}}\PY{p}{]}\PY{o}{.}\PY{n}{hist}\PY{p}{(}\PY{n}{bins}\PY{o}{=}\PY{n}{np}\PY{o}{.}\PY{n}{arange}\PY{p}{(}\PY{l+m+mi}{0}\PY{p}{,} \PY{l+m+mi}{7}\PY{p}{,} \PY{o}{.}\PY{l+m+mi}{5}\PY{p}{)}\PY{p}{)}
        \PY{n}{axis}\PY{o}{.}\PY{n}{set\PYZus{}xlabel}\PY{p}{(}\PY{l+s}{\PYZsq{}}\PY{l+s}{Miles between stations}\PY{l+s}{\PYZsq{}}\PY{p}{)}
        \PY{n}{axis}\PY{o}{.}\PY{n}{set\PYZus{}ylabel}\PY{p}{(}\PY{l+s}{\PYZsq{}}\PY{l+s}{Number of stations}\PY{l+s}{\PYZsq{}}\PY{p}{)}
        \PY{n}{axis}\PY{o}{.}\PY{n}{set\PYZus{}title}\PY{p}{(}\PY{l+s}{\PYZsq{}}\PY{l+s}{Distance Between Subway and Weather Stations}\PY{l+s}{\PYZsq{}}\PY{p}{)}
        \PY{n+nb+bp}{None}
\end{Verbatim}

    \begin{center}
    \adjustimage{max size={0.9\linewidth}{0.9\paperheight}}{Project 1_files/Project 1_19_0.png}
    \end{center}
    { \hspace*{\fill} \\}
    
    The above visualization shows the distance between the weather station
and the subway station it provides data for. Large distances suggest the
weather data is not as representative of the actual weather near the
subway station. Luckily, all of the weather stations are within 6 miles,
with the vast majority being within 2.5 miles.

    \begin{Verbatim}[commandchars=\\\{\}]
{\color{incolor}In [{\color{incolor}10}]:} \PY{c}{\PYZsh{} These calculations are used in the sections below}
         \PY{n}{coord\PYZus{}list} \PY{o}{=} \PY{n}{per\PYZus{}station\PYZus{}coords}\PY{p}{[}\PY{p}{[}\PY{l+s}{\PYZsq{}}\PY{l+s}{latitude}\PY{l+s}{\PYZsq{}}\PY{p}{,} \PY{l+s}{\PYZsq{}}\PY{l+s}{longitude}\PY{l+s}{\PYZsq{}}\PY{p}{]}\PY{p}{]}\PY{o}{.}\PY{n}{values}
         \PY{n}{max\PYZus{}dist} \PY{o}{=} \PY{n+nb}{max}\PY{p}{(}\PY{p}{[}\PY{n}{earth\PYZus{}distance}\PY{p}{(}\PY{p}{[}\PY{n}{x}\PY{p}{[}\PY{l+m+mi}{0}\PY{p}{]}\PY{p}{,} \PY{n}{y}\PY{p}{[}\PY{l+m+mi}{0}\PY{p}{]}\PY{p}{,} \PY{n}{x}\PY{p}{[}\PY{l+m+mi}{1}\PY{p}{]}\PY{p}{,} \PY{n}{y}\PY{p}{[}\PY{l+m+mi}{1}\PY{p}{]}\PY{p}{]}\PY{p}{)}
                         \PY{k}{for} \PY{n}{x} \PY{o+ow}{in} \PY{n}{coord\PYZus{}list} \PY{k}{for} \PY{n}{y} \PY{o+ow}{in} \PY{n}{coord\PYZus{}list}\PY{p}{]}\PY{p}{)}
         \PY{k}{print}\PY{p}{(}\PY{l+s}{\PYZdq{}}\PY{l+s}{Maximum distance between two stations: \PYZob{}:.2f\PYZcb{} miles}\PY{l+s}{\PYZdq{}}\PY{o}{.}\PY{n}{format}\PY{p}{(}\PY{n}{max\PYZus{}dist}\PY{p}{)}\PY{p}{)}
         
         \PY{n}{has\PYZus{}data} \PY{o}{=} \PY{o+ow}{not} \PY{n}{df}\PY{p}{[}\PY{p}{(}\PY{n}{df}\PY{p}{[}\PY{l+s}{\PYZsq{}}\PY{l+s}{station}\PY{l+s}{\PYZsq{}}\PY{p}{]} \PY{o}{==} \PY{l+s}{\PYZsq{}}\PY{l+s}{CYPRESS HILLS}\PY{l+s}{\PYZsq{}}\PY{p}{)}
                           \PY{o}{\PYZam{}} \PY{p}{(}\PY{n}{df}\PY{p}{[}\PY{l+s}{\PYZsq{}}\PY{l+s}{datetime}\PY{l+s}{\PYZsq{}}\PY{p}{]} \PY{o}{==} \PY{l+s}{\PYZsq{}}\PY{l+s}{5/6/11 8:00}\PY{l+s}{\PYZsq{}}\PY{p}{)}\PY{p}{]}\PY{o}{.}\PY{n}{empty}
         \PY{k}{print}\PY{p}{(}\PY{l+s}{\PYZdq{}}\PY{l+s}{Exists data for Cypress Hills at 5/6/11 8:00am: \PYZob{}\PYZcb{}}\PY{l+s}{\PYZdq{}}\PY{o}{.}\PY{n}{format}\PY{p}{(}\PY{n}{has\PYZus{}data}\PY{p}{)}\PY{p}{)}
\end{Verbatim}

    \begin{Verbatim}[commandchars=\\\{\}]
Maximum distance between two stations: 22.01 miles
Exists data for Cypress Hills at 5/6/11 8:00am: False
    \end{Verbatim}

    \subsection*{4. Conclusion}\label{conclusion}

    \emph{4.1 From your analysis and interpretation of the data, do more
people ride the NYC subway when it is raining or when it is not
raining?}

It does not appear that there is a statistically significant
relationship between rain and subway ridership.

\emph{4.2 What analyses lead you to this conclusion? You should use
results from both your statistical tests and your linear regression to
support your analysis.}

The Mann-Whitney U Test between means had a p-value of 7.5\%, which is
too high to be significant. Mann-Whitney U Test performed compared the
mean weekday daily ridership between stations. However, this is
sensitive to noise. If a busy station had a couple of rainy days, the
rainy mean could be brought up significantly. The effect of this could
be reduced if more than a month of data was used.

The ordinary least squares regression also suggested against rain being
a factor. It only appeared to impact global daily ridership by about 16
people, as the coefficient was -15.4. The p-value was also very high,
suggesting that -15.4 was just noise.

    \subsection*{5. Reflection}\label{reflection}

    \emph{5.1 Please discuss potential shortcomings of the methods of your
analysis, including: Dataset, Analysis, such as the linear regression
model or statistical test.}

The dataset used is only from the month of May in 2011. One month is not
a lot of data, a couple of baseball games could throw off these numbers
significantly. This is the largest weakness of this dataset. More data
across a larger frame would greatly increase its usefulness.

There also appears to be missing data points. For example, Cypress Hills
doesn't have any value at 5/6/11 at 8:00am. This throws off calculations
like per-station daily values that were used in the above analysis.

The analysis methods used were powerful but rudimentary. The dependent
variables were picked based on a hunch. Only linear relationships
between variables were considered. More complicated relationships likely
exist. An r-squared value of .56 is good, but a better value is almost
certainly achievable.


    % Add a bibliography block to the postdoc
    
    
    
    \end{document}
